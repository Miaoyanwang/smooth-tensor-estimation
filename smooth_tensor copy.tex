\documentclass[11pt]{article}
\usepackage{lscape}
\usepackage{amsmath,amssymb}
\usepackage{amsthm}
\usepackage{float}
\usepackage{booktabs}
\usepackage{graphicx}
\usepackage{comment}
\usepackage{bm}
\usepackage{gensymb}
\allowdisplaybreaks[4]
\usepackage{geometry}
\geometry{margin=1in}
\usepackage{setspace}
\usepackage{siunitx}
\usepackage{enumitem}
\usepackage{dsfont}

\usepackage{graphics}


\usepackage[utf8x]{inputenc}
\usepackage{bm}

\usepackage{hyperref}
\hypersetup{
    colorlinks=true,
    citecolor = blue,
    linkcolor=blue,
    filecolor=magenta,           
    urlcolor=cyan,
}


\theoremstyle{plain}
\newtheorem{thm}{Theorem}[section]
\newtheorem{lem}{Lemma}
\newtheorem{prop}{Proposition}
\newtheorem{pro}{Property}
\newtheorem{cor}{Corollary}
\newtheorem{ass}{Assumption}

\theoremstyle{definition}
\newtheorem{defn}{Definition}
\newtheorem{exmp}{Example}
\newtheorem{rmk}{Remark}

\usepackage{algpseudocode,algorithm}
\algnewcommand\algorithmicinput{\textbf{Input:}}
\algnewcommand\algorithmicoutput{\textbf{Output:}}
\algnewcommand\INPUT{\item[\algorithmicinput]}
\algnewcommand\OUTPUT{\item[\algorithmicoutput]}



\usepackage[labelfont=bf]{caption}

\setcounter{table}{1}
\usepackage{multirow}
\usepackage{tabularx}

\def\fixme#1#2{\textbf{[FIXME (#1): #2]}}

 

\newcommand*{\KeepStyleUnderBrace}[1]{%f
  \mathop{%
    \mathchoice
    {\underbrace{\displaystyle#1}}%
    {\underbrace{\textstyle#1}}%
    {\underbrace{\scriptstyle#1}}%
    {\underbrace{\scriptscriptstyle#1}}%
  }\limits
}
\usepackage{mathtools}
\mathtoolsset{showonlyrefs=true}


\usepackage{hyperref}
\hypersetup{colorlinks=true}
\usepackage[parfill]{parskip}
\usepackage{bm}
\onehalfspacing

\newcommand{\Hnorm}[1]{\left\lVert#1\right\rVert_{\tH_\alpha}}
%%%%%%%%%%%%%%%%%%%%%%%%%%%%%%%%%%%%%%%%%%%%%%%%%%%%%%%%%%%%%%%%%%%%%
%%             Math Symbols
%%%%%%%%%%%%%%%%%%%%%%%%%%%%%%%%%%%%%%%%%%%%%%%%%%%%%%%%%%%%%%%%%%%%%

%%               Bold Math
\input macros.tex
\def\refer#1{\emph{\color{blue}#1}}
\begin{document}

\begin{center}
{\bf \Large Necessary condition for matrix-valued kernels}\\
Miaoyan Wang, April 26, 2020\\
\vspace{1cm}
\end{center}

\begin{thm}[Necessary condition] Suppose $\mK\colon \mathbb{R}^{d'\times d}\times \mathbb{R}^{d'\times d} \mapsto \mathbb{R}^{d\times d}$ is a function that takes as input a pair of matrices and produces a matrix. Let $\{\mX_i\in\mathbb{R}^{d'\times d}\colon i\in[n]\}$ denote a set of input matrices, and let $\tK$ denote an order-4 $(d,d,n,n)$-dimensional array,
\[
\tK=\entry{\tK(i,i',p,p')},\quad \text{where $\tK(i,i',p,p')$ is the $(p,p')$-th entry of the matrix $\mK(\mX_i, \mX_{i'})$}. 
\]
Then, the factorization $\mK(\mX,\mX')=\mh(\mX)^T\mh(\mX')$ exists for some mapping $\mh$, only if both of the following conditions hold:
\begin{enumerate}
\item [(1)] For every index $i\in[n]$, the slide $\tK(i, i, \cdot,\cdot)\in\mathbb{R}^{d\times d}$ is a symmetric, positive definite matrix.
\item [(2)] For every index $p\in[d]$, the slide $\tK(\cdot,\cdot,p,p)\in\mathbb{R}^{n\times n}$ is a symmetric, positive definite matrix. 

\end{enumerate}
\end{thm}

\bibliographystyle{unsrt}
\bibliography{tensor_wang}

\end{document}
