\documentclass{article}
\usepackage{enumitem}
\usepackage{lipsum}
% Define official colors
\usepackage[dvinames]{xcolor}
%\definecolor{UWcrimson}{RGB}{152,30,50}
\definecolor{UWgray}{HTML}{5e6a71}

\usepackage{svg}
% Measurements
\usepackage[top=2in,left=1.5in,bottom=0.5in,right=0.625in]{geometry}
\usepackage{graphicx}
\usepackage[colorlinks=false,
            pdfborder={0 0 0},
            ]{hyperref}
\usepackage[absolute]{textpos}
\usepackage{ifthen}
\usepackage{soul}


% --- For placement of the horizontal line
\usepackage{tikz}
\usetikzlibrary{calc}

% --- A nice serif font (palatino), but not the prescribed nonfree ITC stone
\usepackage[sc,osf]{mathpazo}
\linespread{1.05}

% Remove paragraph indentation
\parindent0pt
\setlength{\parskip}{0.8\baselineskip}
\raggedright
\pagestyle{empty}
% Ensure consistency in the footer
\urlstyle{sf}

\providecommand\UWfromname{Yazhen Wang}
\providecommand\UWfromtitle{Professor}
\providecommand\UWfromdegree{PhD}
\providecommand\UWinstitution{University of Wisconsin--Madison}
\providecommand\UWfromdept{Department Chair of Statistics}
\providecommand\UWfromaddress{Department of Statistics, Medical Sciences Center, 1300 University Ave Room 1220, Madison, WI 53706.}
\providecommand\UWfromtel{(608) 262-2937}
\providecommand\UWfromemail{\href{mailto:yzwang@stat.wisc.edu}{yzwang@stat.wisc.edu}}
\providecommand\UWtoname{Dear NSF CAREER program officer}
\providecommand\UWtoaddress{Department of XXX
                            USA}
\providecommand\UWdate{\today}
\providecommand\UWopening{Dear \UWtoname,}
\providecommand\UWclosing{Sincerely}
% Update this and the next line to the correct path
%\providecommand\UWsignaturefile{kbroman_sig}
\providecommand\UWlogofile{UWlogo}
\providecommand\UWenclosure{}

\usepackage{fancyhdr}
\pagestyle{fancy}

\setlength{\TPHorizModule}{\paperwidth}\setlength{\TPVertModule}{\paperheight}
\renewcommand{\footrulewidth}{0pt}
\fancyfoot{}
\fancyfoot[L]{%
    {\footnotesize\color{UWgray}\sffamily
    \UWfromaddress\\[-0.1\baselineskip]
   }\color{black}}

\fancypagestyle{firstpagestyle}
{

\fancyhead{}
\fancyhead[CO,CE]{%
  % %\begin{textblock*}{2in}[0.3066,0.39](1.5in,1.33in)
  % \centering \includegraphics[width=2in]{\UWlogofile}
  % % \end{textblock*}
  \begin{textblock}{1}[0.5,0.5](0.5,0.08)
    \includegraphics[width=2in]{\UWlogofile}
  \end{textblock}
     \begin{textblock*}{6.375in}(1.5in,1.75in)   % 6.375=8.5 - 1.5 - 0.625
         \sffamily
         %\hfill \color{UWgray} \UWfromdept
         \hfill \color{UWgray} \UWfromname, \UWfromtitle
                  \\ \hfill \UWfromdept
         \\ \hfill \UWinstitution
                \\ \hfill \UWfromemail
     \end{textblock*}
    % \begin{tikzpicture}[remember picture,overlay]
    %     \draw[color=UWcrimson,line width=0.7pt] (current page.north west)+(1.5in,-1.33in) -- ($(-0.625in,-1.33in)+(current page.north east)$);
    % \end{tikzpicture}
  }
}
\renewcommand{\headrulewidth}{0pt}

\hypersetup{pdfpagemode=UseNone} % don't show bookmarks on initial view

% reformat date as 15 Jan 2015
\usepackage[UKenglish]{datetime}
\newdateformat{UKvardate}{%
\THEDAY\ \shortmonthname[\THEMONTH] \THEYEAR}
\UKvardate


\AtBeginDocument{
    % Text lines should be less than 6in long
    \newgeometry{top=1.2in,left=1in,bottom=1.2in,right=1in}
    \vspace*{1.2in}

    \thispagestyle{firstpagestyle}

    \UWdate
    \bigskip

   % \UWtoname\ifthenelse{\equal{\UWtoname}{}}{}{\\}
   % \UWtoaddress
   % \bigskip

    \UWopening\par
    }

\AtEndDocument{
    \par\vspace{2ex}
    \UWclosing,

    \ifthenelse{\equal{\UWsignaturefile}{}}{\bigskip\bigskip}{\includegraphics[width=1.2in]{\UWsignaturefile}\\[-0.2\baselineskip]}

    \UWfromname \\
    \UWfromtitle\ifthenelse{\equal{\UWfromtitle}{}}{}{\\}
    \UWenclosure
}

\usepackage{ragged2e}
\justifying
\usepackage{amsmath,amssymb}

\begin{document}
This letter is to enthusiastically support Dr.\ Miaoyan Wang's application for the National Science Foundation CAREER Award. Dr.\ Wang joined our department in Fall 2018, and she has been very productive in both research and teaching.  A highlight of Dr.\ Wang's teacher-scholar accomplishment is her two best student paper awards (as advisor) from American Statistical Association in 2021. Dr.\ Wang has proposed an innovative CAREER plan that will strengthen multidisciplinary collaboration while integrating research and educational activities at the interaction of statistics, machine learning, and computer science. 

Dr.\ Wang is one of the most exciting young leaders who meld contributions to both foundations and applications of statistics. She has an exceptional publication track record on making fundamental contribution to statistical learning theory. Equally exceptional is her ability to push the boundary of statistics where scientific question leads, and her development of domain knowledge that expands the core statistical areas. Dr.\ Wang CAREER proposal builds on her established record of research and teaching on statistical machine learning. The proposal spans the full spectrum of foundational data science challenges through the lens of tensor problems. Dr.\ Wang is pushing forward the new application domains and striking out in statistical areas where revolutionized statistical tools are needed. Dr.\ Wang puts forward both of these goals in her CAREER proposal. 

Dr.\ Wang's educational plan to establish a unique research-training experience and broaden participation in STEM are extremely well integrated with her research goals. Dr.\ Wang teaches an undergraduate course on data analytics, as well as an in depth seminar course on statistical machine learning, both of which are highly related to her research. She is a natural born teacher and has been honored Madison Teaching and Learning Excellence (MTLE) Fellow. In addition, Dr.\ Wang mentors several undergraduate students and always has students' best interests at heart. Her 2019 NeurIPS paper with a female undergraduate student has encouraged the student to pursue PhD in top CS program. Many young faculty may investigate less to undergraduate students than to graduate students given the limited resources. However, Dr.\ Wang shows unselfish supports to all of them, sponsors international conference trips, and trains them into next generation of scholars. Her mentoring story has attracted media coverage and popularity among students. 

The educational aspects of Dr.\ Wang's plan are well supported by existing research and training infrastructure both in Dr.\ Wang's group and across the university. Dr.\ Wang plans to establish her group as a unique center for tightly integrated training in two key components: statistical foundation and domain science collaboration. Dr.\ Wang and her students have been actively contributing to Institute for Foundations of Data Science (IFDS), a four-university, multi-disciplinary initiative among the Universities of Washington, Wisconsin-Madison, California Santa Cruz, and Chicago. Students from diverse backgrounds will work with each other in an extremely collaborative, diverse training environment. The multi-disciplinary nature of Dr. Wang publication record so far, the current spread of projects in her group, and the details of the proposed research make this goal very realistic. The Department of Statistics at UW-Madison is deeply committed to supporting the continued success and growth of Dr.\ Wang’s group, facilitating both education and research aspects of her plan.

Dr.\ Wang aims to make her group a welcoming venue for diverse trainees, thereby broadening participation in STEM. Dr.\ Wang shows great leadership in both internal and external committees to facilitate her outreach goals. For example, (1) Dr.\ Wang takes active roles in two international organizations, Women in Probability, and Women in Machine Learning, to improve the visibility of women’s research and mentor women students. (2) Dr.\ Wang participates UW Madison Gooey Chocolate Cake Lunch to promotes interaction and collaboration among women faculty. In addition, Dr.\ Wang plans to make use of two further UW institutions to facilitate her diversity-inclusion goals by partnering with the Data Science Hub and the Division of Diversity, Equity \& Educational Achievement (DDEEA) program in UW-Madison. Dr.\ Wang will take concrete steps toward engaging more diverse undergraduates from underrepresented pools.  

The Department is very strongly invested in young faculty. We recognize the importance of strong mentoring and protection of the PI’s time to allow this already successful faculty member to continue to grow into an internationally recognized leader. Our commitment is detailed below:
\begin{itemize}[wide,labelwidth=!, labelindent=0pt,itemsep=0.1ex,parsep=0ex,topsep=0pt]
\item Mentoring: we provide a mentoring team involving multiple senior faculty members at various career stages with annual meetings focused on topics on mentoring graduate students, teaching, requirements for (early) tenure, outreach, etc. 
\item Support Staff: the Department has an excellent support staff to provide assistance in grant application and management, a computer lab with three dedicated technical support personnel, and other support personnel who oversee administrative functions. 
\item Research Facility: Dr.\ Wang and her students have free access to the Center for High Throughput Computing (CHTC). The CHTC supports a variety of scalable computing resources and services for UW- affiliated researchers and their collaborators, including high-throughput computing, tightly-coupled computations (e.g. message passing interface, or “MPI”), high-memory, and GPUs.
\item Service: Dr.\ Wang is well protected from overloaded service commitments and the Department Chair provides assistance in coordination when needed. We have and will continue to offer flexible teaching/service arrangements when Dr.\ Wang participates semester-long programs out-of-town at Simons Institute, IPAM (Institute for Pure \& Applied Mathematics), IAS (Advanced Research Institute), etc. 
\end{itemize}

Dr.\ Wang is at the height of her research and mentoring abilities, and she has shown great momentum as an independent, young field leader. She has continued to impact the larger community as an outstanding female role model. If I were to bet on a young researcher’s future success and impact, it is hard to imagine a surer bet than Dr.\ Wang. Our Department and University are truly fortunate to have her as a member of the faculty, and we will continue to foster her professional development.

Finally, I confirm that Dr.\ Wang meets the requirements for the CAREER award. She holds a doctoral degree, is in the fourth-year of the tenure-track position, and is currently untenured. She has not previously applied for this award. Dr.\ Wang's U.S. permanent residency is pending as of today; otherwise I would strongly encourage you to consider her for the PECASE award as well.


\end{document}
