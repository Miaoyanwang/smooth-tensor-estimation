\documentclass[11pt]{article}
\usepackage{lscape}
\usepackage{amsmath,amssymb}
\usepackage{amsthm}
\usepackage{float}
\usepackage{booktabs}
\usepackage{graphicx}
\usepackage{comment}
\usepackage{bm}
\usepackage{gensymb}
\allowdisplaybreaks[4]
\usepackage{geometry}
\geometry{margin=1in}
\usepackage{setspace}
\usepackage{siunitx}
\usepackage{enumitem}
\usepackage{dsfont}

\usepackage{graphics}


\usepackage[utf8x]{inputenc}
\usepackage{bm}

\usepackage{hyperref}
\hypersetup{
    colorlinks=true,
    citecolor = blue,
    linkcolor=blue,
    filecolor=magenta,           
    urlcolor=cyan,
}


\theoremstyle{plain}
\newtheorem{thm}{Claim}[section]
\newtheorem{lem}{Lemma}
\newtheorem{prop}{Proposition}
\newtheorem{pro}{Property}
\newtheorem{cor}{Corollary}
\newtheorem{ass}{Assumption}

\theoremstyle{definition}
\newtheorem{defn}{Definition}
\newtheorem{exmp}{Example}
\newtheorem{rmk}{Remark}

\usepackage{algpseudocode,algorithm}
\algnewcommand\algorithmicinput{\textbf{Input:}}
\algnewcommand\algorithmicoutput{\textbf{Output:}}
\algnewcommand\INPUT{\item[\algorithmicinput]}
\algnewcommand\OUTPUT{\item[\algorithmicoutput]}



\usepackage[labelfont=bf]{caption}

\setcounter{table}{1}
\usepackage{multirow}
\usepackage{tabularx}

\def\fixme#1#2{\textbf{[FIXME (#1): #2]}}

 

\newcommand*{\KeepStyleUnderBrace}[1]{%f
  \mathop{%
    \mathchoice
    {\underbrace{\displaystyle#1}}%
    {\underbrace{\textstyle#1}}%
    {\underbrace{\scriptstyle#1}}%
    {\underbrace{\scriptscriptstyle#1}}%
  }\limits
}
\usepackage{mathtools}
\mathtoolsset{showonlyrefs=true}


\usepackage{hyperref}
\hypersetup{colorlinks=true}
\usepackage[parfill]{parskip}
\usepackage{bm}
\onehalfspacing

\newcommand{\Hnorm}[1]{\left\lVert#1\right\rVert_{\tH_\alpha}}
%%%%%%%%%%%%%%%%%%%%%%%%%%%%%%%%%%%%%%%%%%%%%%%%%%%%%%%%%%%%%%%%%%%%%
%%             Math Symbols
%%%%%%%%%%%%%%%%%%%%%%%%%%%%%%%%%%%%%%%%%%%%%%%%%%%%%%%%%%%%%%%%%%%%%

%%               Bold Math
\input macros.tex
\def\refer#1{\emph{\color{blue}#1}}
\begin{document}

\begin{center}
{\bf \Large Smooth tensor estimation and completion}\\
Miaoyan Wang, April 26, 2020
\end{center}
\section{Model}

Let $\tY=\entry{y_{i_1,\ldots,i_K}}\in\{0,1\}^{d \times \cdots \times d}$ be an order-$K$, $(d,\ldots,d)$-dimensional binary tensor. Assume that the entries of $\tY$ are independent Bernoulli random variables with success probabilities
\[
\mathbb{P}\left(y_{i_1,\ldots,i_K}=1\right)=f\left(\xi^{(1)}_{i_1},\ldots,\xi^{(K)}_{i_K}\right),\quad \text{for all } (i_1,\ldots,i_K)\in[d]\times \cdots \times [d],
\]
where $f\colon [0,1]^K\mapsto [0,1]$ is an unknown multivariate function belonging to a function class $f\in \tF_\alpha(M)$, and $\mxi^{(k)}=(\xi^{(k)}_1,\ldots,\xi^{(k)}_d)\in[0,1]^d$ are mutually independent random vectors following  (unknown) distributions $\mathbb{P}_{\mxi^{(k)}}$ for all $k\in[K]$. Furthermore, we define the function class
 \[ 
\tF_\alpha(M)=\{f\colon \text{Im}(f)\in[0,1]\ \text{and}\ \Hnorm{f}\leq M\},
\]
where $\alpha>0$ is the smoothness parameter and $M>0$ is the upper bound of H\"{o}lder norm $\Hnorm{\cdot}$. Specifically, 
\[
\Hnorm{f}\stackrel{\text{def}}{=}\max_{|\mi|\leq \lfloor \alpha \rfloor}\sup_{\mx \in \tD} |\nabla_{\mi} f(\mx)|+\max_{|\mi| =\lfloor \alpha \rfloor}\sup_{\mx\neq \mx' \in \tD} {\nabla_{\mi} |f(\mx)-\nabla_{\mi} f(\mx')|\over \onenorm{\mx-\mx'}^{\alpha-\lfloor \alpha \rfloor}},
\]
where we have used the short-hand notion
\[
 \nabla_{\mi} f(\mx) ={\partial^{i_1+\cdots+i_K}\over \partial x^{i_1}_1\cdots \partial x^{i_K}_K} f(x_1,\ldots,x_k),
\]
for multi-indices $\mi=(i_1,\ldots,i_K)$ with $|\mi|=i_1+\cdots+i_K$, and $\mx=(x_1,\ldots,x_K)$ in the domain.
\bibliographystyle{unsrt}
\bibliography{tensor_wang}

\end{document}
