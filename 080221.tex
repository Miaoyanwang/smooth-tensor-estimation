\documentclass[11pt]{article}
\usepackage{lscape}
\usepackage{amsmath,amssymb}
\usepackage{amsthm}
\usepackage{float}
\usepackage{booktabs}
\usepackage{graphicx}
\usepackage{comment}
\usepackage{bm}
\usepackage{gensymb}
\allowdisplaybreaks[4]
\usepackage{geometry}
\geometry{margin=1in}
\usepackage{setspace}
\usepackage{siunitx}
\usepackage{enumitem}
\usepackage{dsfont}
\usepackage{arydshln}
\usepackage{algorithm}

\newcommand*{\vertbar}{\rule[-1ex]{0.5pt}{2.5ex}}
\newcommand*{\horzbar}{\rule[.5ex]{2.5ex}{0.5pt}}


\usepackage{graphics}
\allowdisplaybreaks

\usepackage[utf8x]{inputenc}
\usepackage{bm}

\usepackage{hyperref}
\hypersetup{
    colorlinks=true,
    citecolor = blue,
    linkcolor=blue,
    filecolor=magenta,           
    urlcolor=cyan,
}


\usepackage{amsthm}
\theoremstyle{definition}
\newtheorem{thm}{Theorem}
\newtheorem{lem}[thm]{Lemma}
\newtheorem{pro}[thm]{Property}
\newtheorem{cor}[thm]{Corollary}
\newtheorem{ass}{Assumption}
\newtheorem{prob}{Problem}
\newtheorem{prop}[thm]{Proposition}
\newtheorem{defn}{Definition}
\newtheorem{exmp}{Example}
\newtheorem{rmk}{Remark}



\usepackage{algpseudocode}

\newcommand\Algphase[1]{%
\vspace*{-.7\baselineskip}\Statex\hspace*{\dimexpr-\algorithmicindent-2pt\relax}\rule{\textwidth}{0.4pt}%
\Statex\hspace*{-\algorithmicindent}\textbf{#1}%
\vspace*{-.7\baselineskip}\Statex\hspace*{\dimexpr-\algorithmicindent-2pt\relax}\rule{\textwidth}{0.4pt}%
}
\algnewcommand\algorithmicinput{\textbf{Input:}}
\algnewcommand\algorithmicoutput{\textbf{Output:}}
\algnewcommand\INPUT{\item[\algorithmicinput]}
\algnewcommand\OUTPUT{\item[\algorithmicoutput]}

\def\sign{\textup{sgn}}
\def\srank{\textup{srank}}
\def\rank{\textup{rank}}
\def\caliP{\mathscr{P}_{\textup{sgn}}}
\def\risk{\textup{Risk}}

\usepackage[labelfont=bf]{caption}

\setcounter{table}{1}
\usepackage{multirow}
\usepackage{tabularx}

\def\fixme#1#2{\textbf{[FIXME (#1): #2]}}

\def\Holder{\text{H\"{o}lder }}

\newcommand*{\KeepStyleUnderBrace}[1]{%f
  \mathop{%
    \mathchoice
    {\underbrace{\displaystyle#1}}%
    {\underbrace{\textstyle#1}}%
    {\underbrace{\scriptstyle#1}}%
    {\underbrace{\scriptscriptstyle#1}}%
  }\limits
}
\usepackage{mathtools}
\mathtoolsset{showonlyrefs=true}

\begingroup
\makeatletter
\@for\theoremstyle:=definition,remark,plain\do{%
\expandafter\g@addto@macro\csname th@\theoremstyle\endcsname{%
\addtolength\thm@preskip\parskip
}%
}
\endgroup


\usepackage{hyperref}
\hypersetup{colorlinks=true}
\usepackage[parfill]{parskip}
\usepackage{bm}
\onehalfspacing

\newcommand{\maxnorm}[1]{\left\lVert#1\right\rVert_{\infty}}
\newcommand{\Hnorm}[1]{\left\lVert#1\right\rVert_{\tH_\alpha}}
\newcommand{\nullnorm}[1]{\left\lVert#1\right\rVert}
\def\trueB{\mB^{\text{true}}}
\def\newX{\mX_{\textup{new}}}
\def\newy{y_{\textup{new}}}
\def\sign{\textup{sign}}
\def\bayesf{f_{\textup{bayes}}}
%%%%%%%%%%%%%%%%%%%%%%%%%%%%%%%%%%%%%%%%%%%%%%%%%%%%%%%%%%%%%%%%%%%%%
%%             Math Symbols
%%%%%%%%%%%%%%%%%%%%%%%%%%%%%%%%%%%%%%%%%%%%%%%%%%%%%%%%%%%%%%%%%%%%%

%%               Bold Math
\input macros.tex
\def\refer#1{\emph{\color{blue}#1}}
\begin{document}

\begin{center}
{\bf \Large Summary of Theory}\\
Miaoyan Wang, Oct 4, 2020\\
\end{center}
\begin{ass}[Lipschitz]
\[
|\Theta(\omega)-\Theta(\omega')|\leq {|\omega-\omega'|\over d},\quad \text{for all }\omega, \omega'\in[d]^m.
 \]
\end{ass}

\begin{ass}[Inverse Lipschitz in indices]
\[
{|i-j|\over d}\leq \text{deg}(i)-\text{deg}(j),\quad\text{for all }i>j \in[d].
\]
\end{ass}

\begin{ass}[Inverse Lipschitz in tensors]
\[
{\FnormSize{}{\Theta(i,\colon)-\Theta(j,\colon)} \over d^{(m-1)/2}} \leq |\text{deg}(i)-\text{deg}(j)|,\quad\text{for all }i, j \in[d].
\]
\end{ass}

Algorithm:
\begin{itemize}
\item Step 1: compute empirical degree $ \widehat{\text{deg}(i)}={1\over d^{m-1}}\tY(i,\mathbf{1},\ldots,\mathbf{1})$.
%\item Step 2: find partition with radius $r:=d^{(m-1)/2$ \\
%Set $\tS=[d]$.\\
%while $\tS\neq \emptyset$
%define neighborhood $\tN_i=\{i'\colon |\widehat{\text{deg}}(i')-\widehat{\text{deg}}(i)|\leq r} \}$.
%$\pi(i')=i$ for all $i'\in\tN_i$
%$\tN\leftarrow \tN/\tN_i$
\item Step 3: define estimate
\[
\hat \Theta(i,\colon)={1\over |\tN_i|}\sum_{i'\in\tN_i}\tY(i',\colon).
\]
\end{itemize}


\begin{algorithm}[h!]
  \caption{Partition of $[d]$ based on critical radius $r$}\label{alg:tensorT}
 \begin{algorithmic}[1] 
 \INPUT a length-$d$ vector $\mv$, and critical radius $r$.
 \OUTPUT a partition $\sigma$ over $[d]$. 
 \State Set $\tS=[d]$ and $r=1$
 \While {$\tS\neq \emptyset$}
 \State Randomly select $i\in\tS$
 \State Find a neighborhood of $i$ based on critical radius $\tN_i:=\{i'\colon |\mv(i)-\mv(i')|\leq r \}$
 \State label $\sigma(i')=r$ for all $i'\in\tN_i$
 \State Update $\tS\leftarrow \tS/\tN_i$ and $r\leftarrow r+1$.
 \EndWhile
 \end{algorithmic}
\end{algorithm}

 
\begin{ass} Define neighborhood 
\[
{1\over d^m}\FnormSize{}{\hat \tP - \tP} \leq  {1\over d^m}\FnormSize{}{\sum_{\omega\in \tN}{\tA(\omega)-\tP(\omega)\over |\tN_i|}
\]
\end{ass}

\bibliographystyle{unsrt}
\bibliography{tensor_wang}

\end{document}
