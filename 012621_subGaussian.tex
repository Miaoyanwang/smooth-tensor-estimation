\documentclass[11pt]{article}
\usepackage{lscape}
\usepackage{amsmath,amssymb}
\usepackage{amsthm}
\usepackage{float}
\usepackage{booktabs}
\usepackage{graphicx}
\usepackage{comment}
\usepackage{bm}
\usepackage{gensymb}
\allowdisplaybreaks[4]
\usepackage{geometry}
\geometry{margin=1in}
\usepackage{setspace}
\usepackage{siunitx}
\usepackage{enumitem}
\usepackage{dsfont}
\usepackage{arydshln}
\usepackage{natbib}

\newcommand*{\vertbar}{\rule[-1ex]{0.5pt}{2.5ex}}
\newcommand*{\horzbar}{\rule[.5ex]{2.5ex}{0.5pt}}


\usepackage{graphics}
\allowdisplaybreaks

\usepackage[utf8x]{inputenc}
\usepackage{bm}

\usepackage{hyperref}
\hypersetup{
    colorlinks=true,
    citecolor = blue,
    linkcolor=blue,
    filecolor=magenta,           
    urlcolor=cyan,
}


\theoremstyle{plain}
\newtheorem{thm}{Theorem}[section]
\newtheorem{lem}{Lemma}
\newtheorem{pro}{Property}
\newtheorem{cor}{Corollary}

\theoremstyle{definition}
\newtheorem{assumption}{Assumption}
\newtheorem{prob}{Problem}
\newtheorem{prop}{Proposition}
\newtheorem{defn}{Definition}
\newtheorem{exmp}{Example}
\newtheorem{rmk}{Remark}
\newtheorem{con}{Conjecture}

\usepackage{algpseudocode,algorithm}
\algnewcommand\algorithmicinput{\textbf{Input:}}
\algnewcommand\algorithmicoutput{\textbf{Output:}}
\algnewcommand\INPUT{\item[\algorithmicinput]}
\algnewcommand\OUTPUT{\item[\algorithmicoutput]}

\def\marginest{\hat \Theta^{\textup{margin}}}
\def\MLEest{\hat \Theta^{\textup{MLE}}}
\def\sign{\textup{sign }}

\usepackage[labelfont=bf]{caption}

\setcounter{table}{1}
\usepackage{multirow}
\usepackage{tabularx}

\def\fixme#1#2{\textbf{[FIXME (#1): #2]}}

\def\Holder{\text{H\"{o}lder }}

\newcommand*{\KeepStyleUnderBrace}[1]{%f
  \mathop{%
    \mathchoice
    {\underbrace{\displaystyle#1}}%
    {\underbrace{\textstyle#1}}%
    {\underbrace{\scriptstyle#1}}%
    {\underbrace{\scriptscriptstyle#1}}%
  }\limits
}
\usepackage{mathtools}
\mathtoolsset{showonlyrefs=true}

\begingroup
\makeatletter
\@for\theoremstyle:=definition,remark,plain\do{%
\expandafter\g@addto@macro\csname th@\theoremstyle\endcsname{%
\addtolength\thm@preskip\parskip
}%
}
\endgroup


\usepackage{hyperref}
\hypersetup{colorlinks=true}
\usepackage[parfill]{parskip}
\usepackage{bm}
\onehalfspacing

\newcommand{\maxnorm}[1]{\left\lVert#1\right\rVert_{\infty}}
\newcommand{\Hnorm}[1]{\left\lVert#1\right\rVert_{\tH_\alpha}}
\newcommand{\nullnorm}[1]{\left\lVert#1\right\rVert}
%%%%%%%%%%%%%%%%%%%%%%%%%%%%%%%%%%%%%%%%%%%%%%%%%%%%%%%%%%%%%%%%%%%%%
%%             Math Symbols
%%%%%%%%%%%%%%%%%%%%%%%%%%%%%%%%%%%%%%%%%%%%%%%%%%%%%%%%%%%%%%%%%%%%%

%%               Bold Math
\input macros.tex
\def\refer#1{\emph{\color{blue}#1}}
\begin{document}

\begin{center}
{\bf \Large Sub-Gaussianity of GLM tensor with bounded variance}\\
Miaoyan Wang, Jan 26, 2021\\
\end{center}


\begin{assumption}[GLM tensor]\label{def:GLM}
We call the tensor $\tY=\entry{y_\omega}$ is exponential family tensor with bounded variance, if the following two assumptions are met. 
\begin{enumerate}[leftmargin=-1pt,topsep=0pt,itemsep=0ex,partopsep=0ex,parsep=0ex]
\item []1. [GLM density] Conditional on canonical parameter tensor $\Theta=\entry{\theta_\omega}$, the tensor entries $y_\omega$'s are independent of each other, and $y_\omega|\theta_\omega$ follows a generalized linear model (GLM) with density
\begin{equation}\label{eq:density}
p(y_\omega|\theta_\omega)=c(y_\omega,\phi)\exp\left(\frac{y_\omega \theta_\omega- b(\theta_\omega)}{\phi}\right),
\end{equation}
where $b(\cdot)$ is a known function depending on the distribution family of $y_\omega$, $\phi>0$ is the dispersion parameter, and $c(\cdot)$ is a known normalizing function.
\item []2. [Boundedness] The parameter tensor $\Theta$ is bounded, i.e, $\mnormSize{}{\Theta}\leq \alpha$ for some constant $\alpha>0$. 
\end{enumerate}
\end{assumption}

\noindent {\bf Proposition 1} (sub-Gaussian residuals under bounded variance).\label{prop}
Let $\tY$ be a GLM data tensor, and $\tE=\tY-b'(\Theta)$ be the residual tensor, where $b'(\cdot)$ denotes the first-order derivative. Under Assumption~\ref{def:GLM}, the entries of $\tE$ are independent sub-Gaussian entries with parameter $(\phi U)$, where $U=\max_{|\theta|\leq \alpha}b''(\theta)<\infty$ and $\phi>0$ is the dispersion parameter in the GLM density.


\begin{proof} It is easy to see that the entries of $\tE=\entry{\varepsilon_\omega}$ are independent conditional on $\Theta=\entry{\theta_\omega}$. Furthermore, we show that $\varepsilon_\omega$ is a sub-Gaussian random variable under the boundedness condition on $\theta_\omega$. For notational convinene, we drop the subscript $\omega$, and simply write $\varepsilon$ and $\theta$. By the definition of sub-Gaussian random variable, it suffices to show 
\[
\mathbb{E}\left[\exp(t\varepsilon|\theta)\right]\leq \exp\left(\phi U t^2\over 2 \right), \quad \text{for all }t\in\mathbb{R}.
\]
By the definition of GLM density, we have
\begin{align}
\mathbb{E}[\exp(t\varepsilon|\theta)]&=\int c(y,\phi) \exp\left({y\theta  - b(\theta)\over \phi}   \right)\exp \left[t(y-b'(\theta))\right]dy\\
&=\int c(y,\phi)\exp \left( {y(\theta + \phi t) - b (\theta+\phi t)+b(\theta+\phi t)-b(\theta)-\phi t b'(\theta) \over \phi}\right)dy\\
&=\exp\left( {b(\theta+\phi t)-b(\theta)-\phi t b'(\theta) \over \phi} \right)\\
&\leq \exp\left(\phi U t^2\over 2 \right),
\end{align}
where the last inequality follows from Taylor expansion and the definition of $U$. Therefore, $\varepsilon$ is sub-Gaussian-$(\phi U)$. 
\end{proof}


\bibliographystyle{plainnat}
\bibliography{tensor_wang}
\end{document}
