\documentclass[11pt]{article}
\usepackage{lscape}
\usepackage{amsmath,amssymb}
\usepackage{amsthm}
\usepackage{float}
\usepackage{booktabs}
\usepackage{graphicx}
\usepackage{comment}
\usepackage{bm}
\usepackage{gensymb}
\allowdisplaybreaks[4]
\usepackage{geometry}
\geometry{margin=1in}
\usepackage{setspace}
\usepackage{siunitx}
\usepackage{enumitem}
\usepackage{dsfont}
\usepackage{arydshln}
\usepackage{natbib}

\newcommand*{\vertbar}{\rule[-1ex]{0.5pt}{2.5ex}}
\newcommand*{\horzbar}{\rule[.5ex]{2.5ex}{0.5pt}}


\usepackage{graphics}
\allowdisplaybreaks

\usepackage[utf8x]{inputenc}
\usepackage{bm}

\usepackage{hyperref}
\hypersetup{
    colorlinks=true,
    citecolor = blue,
    linkcolor=blue,
    filecolor=magenta,           
    urlcolor=cyan,
}


\theoremstyle{plain}
\newtheorem{thm}{Theorem}[section]
\newtheorem{lem}{Lemma}
\newtheorem{pro}{Property}
\newtheorem{cor}{Corollary}

\theoremstyle{definition}
\newtheorem{assumption}{Assumption}
\newtheorem{prob}{Problem}
\newtheorem{prop}{Proposition}
\newtheorem{defn}{Definition}
\newtheorem{exmp}{Example}
\newtheorem{rmk}{Remark}
\newtheorem{con}{Conjecture}

\usepackage{algpseudocode,algorithm}
\algnewcommand\algorithmicinput{\textbf{Input:}}
\algnewcommand\algorithmicoutput{\textbf{Output:}}
\algnewcommand\INPUT{\item[\algorithmicinput]}
\algnewcommand\OUTPUT{\item[\algorithmicoutput]}

\def\marginest{\hat \Theta^{\textup{margin}}}
\def\MLEest{\hat \Theta^{\textup{MLE}}}
\def\sign{\textup{sign }}

\usepackage[labelfont=bf]{caption}

\setcounter{table}{1}
\usepackage{multirow}
\usepackage{tabularx}

\def\fixme#1#2{\textbf{[FIXME (#1): #2]}}

\def\Holder{\text{H\"{o}lder }}

\newcommand*{\KeepStyleUnderBrace}[1]{%f
  \mathop{%
    \mathchoice
    {\underbrace{\displaystyle#1}}%
    {\underbrace{\textstyle#1}}%
    {\underbrace{\scriptstyle#1}}%
    {\underbrace{\scriptscriptstyle#1}}%
  }\limits
}
\usepackage{mathtools}
\mathtoolsset{showonlyrefs=true}

\begingroup
\makeatletter
\@for\theoremstyle:=definition,remark,plain\do{%
\expandafter\g@addto@macro\csname th@\theoremstyle\endcsname{%
\addtolength\thm@preskip\parskip
}%
}
\endgroup


\usepackage{hyperref}
\hypersetup{colorlinks=true}
\usepackage[parfill]{parskip}
\usepackage{bm}
\onehalfspacing

\newcommand{\maxnorm}[1]{\left\lVert#1\right\rVert_{\infty}}
\newcommand{\Hnorm}[1]{\left\lVert#1\right\rVert_{\tH_\alpha}}
\newcommand{\nullnorm}[1]{\left\lVert#1\right\rVert}
%%%%%%%%%%%%%%%%%%%%%%%%%%%%%%%%%%%%%%%%%%%%%%%%%%%%%%%%%%%%%%%%%%%%%
%%             Math Symbols
%%%%%%%%%%%%%%%%%%%%%%%%%%%%%%%%%%%%%%%%%%%%%%%%%%%%%%%%%%%%%%%%%%%%%

%%               Bold Math
\input macros.tex
\def\refer#1{\emph{\color{blue}#1}}
\begin{document}

\begin{center}
{\bf \Large Compatibility of smooth tensor and low-rank tensor}\\
Miaoyan Wang, July 25, 2021\\
\end{center}

Let $\mS^r=\{\mx\in\mathbb{R}^r\colon \vnormSize{}{\mx}\leq 1\}$ be an $r$-dimensional unit ball. We also use $\mx=(\mx(1),\ldots,\mx(r))^T$ to denote the vector in $\mathbb{R}^r$, where $\mx(i)\in\mathbb{R}$ denotes $i$-th element in the vector. Consider the following generative process: 
\begin{enumerate}[wide]
\item[] Step 1. Fix an $m$-variate function
\begin{align}
f\colon \mS^{r}\times\cdots\times \mS^{r} &\to [0,1],\\
(\mx_1,\ldots,\mx_m)&\mapsto f(\mx_1,\ldots,\mx_m)=\sum_{i=1}^r\mx_1(i)\cdots\mx_m(i).
\end{align}
By the multilinearlity of $f$ and boundedness of $\mS^r$, $f$ is a 1-Lipschitz function. 
\item[] Step 2. Draw either random or fixed samples $\mx_1,\ldots,\mx_d$ from $\mS^r$. 
\item[] Step 3. Define the signal tensor 
\[
\Theta(i_1,\ldots,i_m)=f(\mx_{i_1},\ldots,\mx_{i_m}), \quad \text{for all }(i_1,\ldots,i_m)\in[d]^m.
\]
By construction, $\Theta$ is a rank-$r$ tensor. 
\end{enumerate}

\begin{thm}[Block approximation to low-rank tensors] Let $\Theta\in[0,1]^m$ be an order-$m$ rank-$r$ tensor generated from steps 1-3. Then, for every $K\in\mathbb{N}_{+}$, there exists a block-$K$ tensor $\bar \Theta$ such that
\[
\FnormSize{}{\Theta-\bar \Theta}^2 \lesssim {d^m\over K^{2/r}}. 
\]
\end{thm}
\begin{proof}
Because $\mS^r$ is a compact set, $\mS^r$ can be covered by $K$ Euclidean balls with radius $K^{-1/r}$. Let $\my_1,\ldots,\my_{K}$ be the center of these balls. Without loss of generality, assume $\my_k\in\mS^r$ for all $k\in[K]$. Then, by definition of covering sets, for every $i\in[d]$, there exists $k=k(i)\in\{1,\ldots,K\}$ such that
\[
\vnormSize{}{\mx_i-\my_{k(i)}}\lesssim {1\over K^{1/r}}. 
\]
By the smoothness of $f$, we have
\begin{equation}\label{eq:f}
|f(\mx_{i_1},\ldots,\mx_{i_m})-f(\my_{k(i_1)},\ldots,\my_{k(i_m)})|^2 \lesssim {1\over K^{2/r}}.
\end{equation}
Notice that there are in total $K$ distinct points $\my_{k(i)}$. Therefore, the tensor $\bar \Theta$ defined by
\[
\bar \Theta(i_1,\ldots,i_m):=f(\my_{k(i_1)},\ldots,\my_{k(i_m)})
\]
has at most $K$ blocks on each model. Based on~\eqref{eq:f}, we have constructed a block-$K$ tensor $\bar \Theta$ such that
\[
\FnormSize{}{\Theta-\bar \Theta}^2 \leq {d^m\over K^{2/r}}. 
\]
\end{proof}
\end{document}