\documentclass[11pt]{article}
\usepackage{lscape}
\usepackage{amsmath,amssymb}
\usepackage{amsthm}
\usepackage{float}
\usepackage{booktabs}
\usepackage{graphicx}
\usepackage{comment}
\usepackage{bm}
\usepackage{gensymb}
\allowdisplaybreaks[4]
\usepackage{geometry}
\geometry{margin=1in}
\usepackage{setspace}
\usepackage{siunitx}
\usepackage{enumitem}
\usepackage{dsfont}

\usepackage{graphics}


\usepackage[utf8x]{inputenc}
\usepackage{bm}

\usepackage{hyperref}
\hypersetup{
    colorlinks=true,
    citecolor = blue,
    linkcolor=blue,
    filecolor=magenta,           
    urlcolor=cyan,
}


\theoremstyle{plain}
\newtheorem{thm}{Theorem}[section]
\newtheorem{lem}{Lemma}
\newtheorem{prop}{Proposition}
\newtheorem{pro}{Property}
\newtheorem{cor}{Corollary}
\newtheorem{ass}{Assumption}

\theoremstyle{definition}
\newtheorem{defn}{Definition}
\newtheorem{exmp}{Example}
\newtheorem{rmk}{Remark}

\usepackage{algpseudocode,algorithm}
\algnewcommand\algorithmicinput{\textbf{Input:}}
\algnewcommand\algorithmicoutput{\textbf{Output:}}
\algnewcommand\INPUT{\item[\algorithmicinput]}
\algnewcommand\OUTPUT{\item[\algorithmicoutput]}



\usepackage[labelfont=bf]{caption}

\setcounter{table}{1}
\usepackage{multirow}
\usepackage{tabularx}

\def\fixme#1#2{\textbf{[FIXME (#1): #2]}}

 

\newcommand*{\KeepStyleUnderBrace}[1]{%f
  \mathop{%
    \mathchoice
    {\underbrace{\displaystyle#1}}%
    {\underbrace{\textstyle#1}}%
    {\underbrace{\scriptstyle#1}}%
    {\underbrace{\scriptscriptstyle#1}}%
  }\limits
}
\usepackage{mathtools}
\mathtoolsset{showonlyrefs=true}


\usepackage{hyperref}
\hypersetup{colorlinks=true}
\usepackage[parfill]{parskip}
\usepackage{bm}
\onehalfspacing

\newcommand{\Hnorm}[1]{\left\lVert#1\right\rVert_{\tH_\alpha}}
\newcommand{\onenormSize}[1]{\left\lVert#1\right\rVert_{_1}}
%%%%%%%%%%%%%%%%%%%%%%%%%%%%%%%%%%%%%%%%%%%%%%%%%%%%%%%%%%%%%%%%%%%%%
%%             Math Symbols
%%%%%%%%%%%%%%%%%%%%%%%%%%%%%%%%%%%%%%%%%%%%%%%%%%%%%%%%%%%%%%%%%%%%%

%%               Bold Math
\input macros.tex
\def\refer#1{\emph{\color{blue}#1}}
\begin{document}
\begin{center}
{\Large \bf Identifiability for sparse Tucker-1 models}

Miaoyan Wang, Oct 1, 2020\\
\vspace{1cm}
\end{center}
\begin{thm}[Identifiability]
Let $\Omega_k\in\mathbb{R}^{d\times d}$ denote the precision matrix for tissue $k\in[K]$. Assume that $\{\Omega_k\}$ admits the rank-$r$ Tucker-1 decomposition model with $r\leq \min(K,d^2)$, 
\begin{equation}\label{eq:1}
\Omega_k=\Theta_0+\sum_{l=1}^r u_{lk}\Theta_l, \quad \text{for all } \quad k=1,\ldots,K.
\end{equation}
Define the vector $\mathbf{u}_l=(u_{l1},\ldots,u_{lK})^T\in\mathbb{R}^K$ and $\text{Supp}(\mathbf{u}_l)=\{i\in[K]\colon \mu_{lk}\neq 0\}$. Suppose the vectors $\{\mathbf{u}_l\}$ and $\Theta_0$ satisfy the following conditions,
\begin{enumerate}
\item The vectors $\{\mathbf{u}_l\}$ have mutually non-overlapping supports; i.e, $\text{Supp}(\mathbf{u}_l)\cap \text{Supp}(\mathbf{u}_l')=0$, for all $l\neq l'\in[r]$. 
\item The vectors $\{\mathbf{u}_l\}$ are normalized such that $\mathbf{u}^T_l\mathbf{u}_l=1$ for all $l\in[r]$.
\item ${1\over n}\sum_k\Omega_k = \Theta_0$.
\end{enumerate}
Then the decomposition~\eqref{eq:1} is unique. Specifically, if $\{\Omega_k\}$ admits another rank-$r$ decomposition that satisfying the above conditions 1-3,
\begin{equation}\label{eq:1}
\Omega_k=\Theta'_0+\sum_{l=1}^r u'_{lk}\Theta'_l, \quad \text{for all } \quad k=1,\ldots,K,
\end{equation}
then we must have $\Theta_l=\Theta'_l$ for $l=0,1,\ldots,r$, and $\mathbf{u}_l=\pm \mathbf{u'}_{\pi(l)}$ up to sign and permutation $\pi:[r]\to[r]$.
\end{thm}


\end{document}
